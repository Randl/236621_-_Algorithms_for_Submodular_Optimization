\section{Introduction to matroids}
Matroid is a basic concept in combinatorial optimization. It was first defined by \citet{whitney1935abstract}. 
\begin{definition}[matroid]
	Matroid $\mathcal{M}$ is a pair $(E,\mathcal{I})$. $E$ is is a finite set (called the ground set) and $\mathcal{I}\neq \emptyset$ is a family of subsets of $E$ (called the independent sets) with the following properties:
	\begin{enumerate}
		\item If $Y\in \mathcal{I}$ then for all $X\subseteq Y$, $X\in \mathcal{I}$.
		\item If $X,Y\in \mathcal{I}$ and $\abs{Y}>\abs{X}$, then exists $e\in Y\setminus X$, $X\cup \qty{e} \int \mathcal{I}$.
	\end{enumerate}
\end{definition}

\paragraph{Notes} All maximal independent sets have same size. Those sets are called basis.


\paragraph{Examples}
\subparagraph{Uniform manifold} 
\begin{align}
	\mathcal{M}_{k} = \qty(E, \qty\bigg{X\subseteq E | \abs{X} \leq k})
\end{align}
\subparagraph{Linear manifold} 
Let $A\in \mathbb{R}^{m\times n}$ be a matrix. Let $E$ be a set of columns of $A$. The set $X\subseteq E$ is independent if its elements are independent. Alternatively, for sub-matrix $A_X$ consisting of columns of $A$:
\begin{align}
\mathcal{I} = \qty{ X\subset E | \rank(A_x) = \abs{X}}
\end{align}


\subparagraph{Graphic matroids} 
Let $G=\qty(V_G,E_G)$ be a graph, $E=E_G$ and
\begin{align}
\mathcal{I} = \qty{X\subseteq E_G | X \text{ is forest}}
\end{align}
\begin{prop}
	$M=(E_G,\mathcal{I})$ is matroid.
\end{prop}
The basis is then spanning trees (or forests if graph is not connected).

\subparagraph{Partition matroid}
For a set $E$ let $E_1,\dots E_k$ be some partition of $E$. Then
\begin{align}
\mathcal{I}= \qty\big{ X\subseteq E \big\vert \forall i=1..k \: \abs{X}\cup E_i  \leq 1  }
\end{align}
\begin{prop}
	$M=(E,\mathcal{I})$ is matroid.
\end{prop}
Note that partition matroid encodes constraints of submodular welfare problem.

Constraint of matching in the bipartite graph can be defined as intersection of two partition matroids.

\begin{definition}[Circuit]
	Circuit in matroid $M=(E,\mathcal{I})$ is a dependent set $X$ such that $X\notin \mathcal{I}$ and for all $x\in X$ $X\setminus \qty{x} \in \mathcal{I}$.
\end{definition}
\begin{definition}[Rank function]
For matroid $M=(E,\mathcal{I})$ rank function $r: 2^{\mathbb{E}} \to \mathbb{N}$ is defined as
\begin{align}
r(x) = \max \qty\big{\abs{Y} \big| Y\subset{X}, Y\in \mathcal{I}}
\end{align}
\end{definition}
\begin{definition}[Rank of matroid]
For matroid $M=(E,\mathcal{I})$ rank of matroid is $\rank(E)$.
\end{definition}
\begin{prop}
	Rank of matroid is submodular function.
\end{prop}


\begin{algorithm}
	\caption{}\label{alg:matroid_basis}
	\begin{algorithmic}[1]
		\Procedure{Greedy}{$E$, $I$}
		\State $S \gets \emptyset$
		\For {$e\in E$}
		\If {$S\cup \qty{x} \in \mathcal{I}$}
		\State $S\gets S\cup \qty{x}$
		\EndIf
		\EndFor
		\State \textbf{return} $S$
		\EndProcedure
	\end{algorithmic}
\end{algorithm}

\begin{prop}
	\vref{alg:matroid_basis} returns basis of $E$.
	\begin{proof}
		Assume $S$ is not a basis and let $B$ be a basis. Exists $x\in B\setminus S$ such that $S\cup \qty{x}$ is independent. However, since we have not added $x$ to $S$, it got to be dependent with $S$.
	\end{proof}
\end{prop}

\paragraph{Question} Given matroid over $E$ (via independence oracle), let weight function $w: E\to \mathbb{R}$ and weight of set be $w(X) = \sum_{x\in X} w(x)$. We want to find independent set (pr basis) of maximal weight.


\begin{algorithm}
	\caption{}\label{alg:matroid_weighted_basis}
	\begin{algorithmic}[1]
		\Procedure{Greedy}{$E$, $I$}
		\State $S \gets \emptyset$
		\For {$e\in E$ from heaviest to lightest}
		\If {$S\cup \qty{x} \in \mathcal{I}$}
		\State $S\gets S\cup \qty{x}$
		\EndIf
		\EndFor
		\State \textbf{return} $S$
		\EndProcedure
	\end{algorithmic}
\end{algorithm}
\begin{prop}
	\vref{alg:matroid_weighted_basis} solves the problem of maximal weight basis.
	\begin{proof}
		We know that for $k = \rank(M)$, the size of the output of algorithm is $k$ and so is size of optimal solution $S^*$.
		Lets assume $S$ is not optimal, thus exists $i$ such that $w(e_i^*) > w(e_i)$.
		
		At iteration at which we added $e_i$ to $S$. At this iteration $\abs{S}=i-1$. take a look at first $i$ elements of $S^*$: this is independent set, and from definition of matroid, exists $e'$ such that $S$ is independent with $e'$. However, $w(e') \geq w(e^*-i) > w(e_i)$, thus we should have added it beforehand.
	\end{proof}
\end{prop}

\section{Continuous extensions of submodular functions} 
\begin{definition}[Continuous extensions of function]
	For $f: 2^N \to \mathbb{R}$ extension of $f$ is $F: [0,1]^N \to \mathbb{E}$ such that for all $S\in N$ (denote by $\mathds{1}_S \in \qty{0,1}^N$ indicator of $S$) $F(\mathds{1}_S) = f(S)$.
\end{definition}

There exist many extensions of submodular functions. In particular, there exist convex and concave extensions of submodular functions. 
\begin{prop}
	Evaluating concave extensions of submodular function in some point is NP-hard.
\end{prop}