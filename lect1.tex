\section{Introduction}
We are looking on $f: 2^N \to \mathbb{R}$ for some set $N = \qty{1, \dots n }$
\begin{definition}
	$f$ is submodular if 
	\begin{align}
	f(A)+f(B) \geq f(A\cap B) + f(A\cup B)
	\end{align}
\end{definition}

\begin{definition}
	Return of $u$ wrt $A$ is $f(A \cup \qty{u }) - f(A)$
	
\end{definition}
\begin{definition}[Diminishing returns]
$f$ has diminishing returns if for $A\subseteq B$
\begin{align}
f(A \cup \qty{u }) - f(A) \geq f(B \cup \qty{u }) - f(B)
\end{align}
\end{definition}

\begin{prop}
	$f$ is submodular iff $f$ has diminishing returns
	\begin{proof}
		$\Rightarrow$:
		
		Let $A\subseteq B \subseteq N$ and $u\notin B$. Lets use submodularity property on $A\cup \qty{ u }$ and $B$:
		\begin{align}
		f(A\cup \qty{ u }) + f(B) \geq f\qty(A\cup \qty{ u } \cup B) + f\qty(9A\cup \qty{ u }0 \cap B)  = f\qty(B\cup \qty{ u }) + f(B)  
		\end{align}
		Thus
		
		\begin{align}
		f(A\cup \qty{ u }) \geq f\qty(B\cup \qty{ u })  
		\end{align}
	\end{proof}


$\Leftarrow$:

We'll proof by induction over $\abs{A\cup B} - \abs{A\cap B}$, i.e., size of symmetric difference.

Basis: $\abs{A\cup B} - \abs{A\cap B} = 0$, then $A=B$, and then submodular property is fulfilled.

Step: assume $\abs{A\cup B} - \abs{A\cap B} = k$. WLOG let $u\in A$ such that $u\notin B$.
\begin{align}
f(A) + f(B) = f(A) - f(A \setminus \qty{u}) + f(A \setminus  \qty{u}) + f(B) \geq\\\geq  f(A) - f(A \setminus \qty{u}) + f(A \setminus  \qty{u} \cup B) + f(A \setminus  \qty{u}  \cap B) \geq\\\geq f(A\cup B) -  f(A\cup B \setminus \qty{ u }) + f(A  \cup B \setminus  \qty{u} ) + f(A  \cap B) = f(A\cup B) + f(A\cap B)
\end{align}
\end{prop}

\begin{definition}[Monotonous function]
	$f$ is non-decreasing monotonous if $\forall A\subseteq B \subseteq N$, $f(A) \leq f(B)$.
\end{definition}

\begin{definition}[Symmetric function]
	$f$ is symmetric if $\forall S \subseteq N$, $f(S) \leq f(N\setminus S)$.
\end{definition}
\begin{definition}[Normalized function]
$f$ is normalized if $f(\emptyset) = 0$.
\end{definition}

\paragraph{Examples}
\subparagraph{Linear function}
$\forall n\in N$ exists weight $w_n$ and 
\begin{align}
f(S) = \sum_{u \in S} w_u + b
\end{align} 

Such $f$ is submodular.
\subparagraph{Budget additive function (clipped linear function)}
$\forall n\in N$ exists weight $w_n$ and 
\begin{align}
f(S) = \min \qty{ \sum_{u \in S} w_u,  b }
\end{align} 

Such $f$ is submodular.

\subparagraph{Coverage function}
Given set $X$ and $n$ subsets $S_1, S_2, \dots, S_n \subset X$ define
\begin{align}
f(S) = \abs{\bigcup_{i \in S} S_i}
\end{align}

This $f$ is obviously submodular.

\subparagraph{Graph cuts}
Let $G+(V,E)$ be a graph and $w: E\to \mathbb{R}^+$ weights of edges. Given a cut $S\subseteq V$ define $\delta(S)$ to be sum of weights of all edges going through the cut. $\delta : 2^V \to \mathbb{R}^+$ is submodular, normalized, and symmetric.


\subparagraph{Rank function}
Let $v_1, \dots, v_n \in \mathbb{R}^d$ vectors, and 

\begin{align}
f(S) = \rank (S) = \dim \text{span} \qty\big( \qty{ v_i | i \in S} )
\end{align}


\section{Submodular optimization}
Given world $N$, submodular function $f: 2^N \to \mathbb{R}^+$, and a family of feasible solutions $\mathcal{I} \subseteq 2^N$
\begin{align}
\max \:& f(S)\\
\st & S\in \mathcal{I}
\end{align}

\paragraph{Note} Most of submodular functions (except for logarithm of determinant of submatrix) are nonnegative. We use the condition to have properly defined multiplicative approximation.
\paragraph{Note} How $f$ is given in input? Obviously, not as a list of values, since it's exponential in $\abs{N}$. Thus we represent $f$ with black box, and same applies for constraints. Usually, constraints are simple. 

\subsection{Examples of submodular optimization problems}
\paragraph{Example}
$f$ is submodular and there are no constraints. It generalizes \textsc{max-cut}, \textsc{max-dicut}
\paragraph{Example}
$f$ is submodular and there is size constraint:

\begin{align}
\max \: & f(S)\\
\st & \abs{S} \leq k
\end{align}
. It generalizes \textsc{max-k-cover}.

\paragraph{Submodular welfare}