\documentclass[]{article}
\usepackage{amsmath}
\usepackage{amsfonts}
\usepackage{amssymb}
\usepackage{gensymb}
\usepackage{graphicx}
\usepackage{svg}
\usepackage{bbding}
\usepackage{mathtools}
\usepackage{centernot} % not parallel, etc.
\usepackage{lmodern}
\usepackage{morewrites}
\usepackage{xcolor,sectsty} % colorful sections
\usepackage[left=10mm, top=10mm, right=10mm, bottom=20mm, nohead]{geometry}
%\usepackage{bigints}
\usepackage{dsfont} %mathbb 1
\usepackage{esint} % beatiful integrals
\usepackage{physics}
\usepackage{amsthm} % theorems
\usepackage{algorithm}
\usepackage{algpseudocode}
\usepackage{algorithmicx}
\usepackage{natbib}


\usepackage{varioref} % references to show page
\usepackage[pagebackref]{hyperref}
\usepackage{url}
\usepackage[capitalise]{cleveref} 

\newcommand{\yrcite}[1]{\citeyearpar{#1}}
\renewcommand{\cite}[1]{\citep{#1}}

\usepackage[T1]{fontenc}
% Nicer default font (+ math font) than Computer Modern for most use cases
% \usepackage{mathpazo} % problems with greek vectors
\usepackage[utf8x]{inputenc} % Allow utf-8 characters in the tex document
% Prevent overflowing lines due to hard-to-break entities
\sloppy 
% Colors for the hyperref package
\definecolor{urlcolor}{rgb}{0,.145,.698}
\definecolor{linkcolor}{rgb}{.71,0.21,0.01}
\definecolor{citecolor}{rgb}{.12,.54,.11}
% Setup hyperref package
\hypersetup{
	breaklinks=true,  % so long urls are correctly broken across lines
	colorlinks=true,
	urlcolor=urlcolor,
	linkcolor=linkcolor,
	citecolor=citecolor,
}


\renewcommand*{\backref}[1]{} % for backref < 1.33 necessary
\renewcommand*{\backrefalt}[4]{%
	\ifcase #1 %
	%
	\or
	(cited on p. #2)%
	\else
	(cited on pp. #2)%
	\fi
}


\DeclareFontFamily{OMX}{lmex}{}
\DeclareFontShape{OMX}{lmex}{m}{n}{<-> lmex10}{}


%colors of sections
\definecolor{secfont}{RGB}{46,116,181}
\definecolor{subfont}{RGB}{146,23,57}
\definecolor{parfont}{RGB}{19,127,43}
\definecolor{subparfont}{RGB}{7,11,100}

\subsectionfont{\color{subfont}}
\sectionfont{\color{secfont}}
\paragraphfont{\color{parfont}}
\subparagraphfont{\color{subparfont}}


% declare a new theorem style
\newtheoremstyle{bluestyle}%
{3pt}% Space above
{3pt}% Space below 
{}% Body font
{}% Indent amount
{\bfseries\color{blue}}% Theorem head font
{.}% Punctuation after theorem head
{.5em}% Space after theorem head
{}% Theorem head spec (can be left empty, meaning ‘normal’)
% declare a new theorem style
\newtheoremstyle{redstyle}{3pt}{3pt}{}{}{\bfseries\color{red}}{.}{.5em}{}
\newtheoremstyle{olivestyle}{3pt}{3pt}{}{}{\bfseries\color{olive}}{.}{.5em}{}
\newtheoremstyle{orangestyle}{3pt}{3pt}{}{}{\bfseries\color{orange}}{.}{.5em}{}
\newtheoremstyle{magentastyle}{3pt}{3pt}{}{}{\bfseries\color{magenta}}{.}{.5em}{}

\theoremstyle{bluestyle}
\newtheorem{theorem}{Theorem}[section]
\theoremstyle{redstyle}
\newtheorem{definition}{Definition}[section]
\theoremstyle{magentastyle}
\newtheorem{coll}{Collary}[theorem]
\theoremstyle{orangestyle}
\newtheorem{lemma}[theorem]{Lemma}
\theoremstyle{olivestyle}
\newtheorem{prop}[theorem]{Proposition}

\newcommand{\st}{\text{s.t.} \:}

%\usepackage{babel}[english]
%opening
\title{ 	236621 - Algorithms for Submodular Optimization}
\author{Roy Schwartz}
% njc860@gmail.com
% Amado 707
% https://sites.google.com/site/njcrawfordacademic/
% David Williams Probability with Martingales
% Rick Derret Probability Theory and examples


\parindent=0em
\begin{document}


\maketitle

\begin{abstract}

\end{abstract}

%\tableofcontents
\section{Introduction}
We are looking on $f: 2^N \to \mathbb{R}$ for some set $N = \qty{1, \dots n }$
\begin{definition}
	$f$ is submodular if 
	\begin{align}
	f(A)+f(B) \geq f(A\cap B) + f(A\cup B)
	\end{align}
\end{definition}

\begin{definition}
	Return of $u$ wrt $A$ is $f(A \cup \qty{u }) - f(A)$
	
\end{definition}
\begin{definition}[Diminishing returns]
$f$ has diminishing returns if for $A\subseteq B$
\begin{align}
f(A \cup \qty{u }) - f(A) \geq f(B \cup \qty{u }) - f(B)
\end{align}
\end{definition}

\begin{prop}
	$f$ is submodular iff $f$ has diminishing returns
	\begin{proof}
		$\Rightarrow$:
		
		Let $A\subseteq B \subseteq N$ and $u\notin B$. Lets use submodularity property on $A\cup \qty{ u }$ and $B$:
		\begin{align}
		f(A\cup \qty{ u }) + f(B) \geq f\qty(A\cup \qty{ u } \cup B) + f\qty((A\cup \qty{ u }) \cap B)  = f\qty(B\cup \qty{ u }) + f(A)  
		\end{align}
		Thus
		
		\begin{align}
		f(A\cup \qty{ u })-f(A) \geq f\qty(B\cup \qty{ u })-f(B)
		\end{align}
	\end{proof}


$\Leftarrow$:

We'll proof by induction over $\abs{A\cup B} - \abs{A\cap B}$, i.e., size of symmetric difference.

Basis: $\abs{A\cup B} - \abs{A\cap B} = 0$, then $A=B$, and then submodular property is fulfilled.

Step: assume $\abs{A\cup B} - \abs{A\cap B} = k$. WLOG let $u\in A$ such that $u\notin B$.
\begin{align}
f(A) + f(B) = f(A) - f(A \setminus \qty{u}) + f(A \setminus  \qty{u}) + f(B) \geq\\\geq  f(A) - f(A \setminus \qty{u}) + f(A \setminus  \qty{u} \cup B) + f(A \setminus  \qty{u}  \cap B) \geq\\\geq f(A\cup B) -  f(A\cup B \setminus \qty{ u }) + f(A  \cup B \setminus  \qty{u} ) + f(A  \cap B) = f(A\cup B) + f(A\cap B)
\end{align}
\end{prop}

\begin{definition}[Monotonous function]
	$f$ is non-decreasing monotonous if $\forall A\subseteq B \subseteq N$, $f(A) \leq f(B)$.
\end{definition}

\begin{definition}[Symmetric function]
	$f$ is symmetric if $\forall S \subseteq N$, $f(S) \leq f(N\setminus S)$.
\end{definition}
\begin{definition}[Normalized function]
$f$ is normalized if $f(\emptyset) = 0$.
\end{definition}

\paragraph{Examples}
\subparagraph{Linear function}
$\forall n\in N$ exists weight $w_n$ and 
\begin{align}
f(S) = \sum_{u \in S} w_u + b
\end{align} 

Such $f$ is submodular.
\subparagraph{Budget additive function (clipped linear function)}
$\forall n\in N$ exists weight $w_n$ and 
\begin{align}
f(S) = \min \qty{ \sum_{u \in S} w_u,  b }
\end{align} 

Such $f$ is submodular.

\subparagraph{Coverage function}
Given set $X$ and $n$ subsets $S_1, S_2, \dots, S_n \subset X$ define
\begin{align}
f(S) = \abs{\bigcup_{i \in S} S_i}
\end{align}

This $f$ is obviously submodular.

\subparagraph{Graph cuts}
Let $G=(V,E)$ be a graph and $w: E\to \mathbb{R}^+$ weights of edges. Given a cut $S\subseteq V$ define $\delta(S)$ to be sum of weights of all edges going through the cut. $\delta : 2^V \to \mathbb{R}^+$ is submodular, normalized, and symmetric.


\subparagraph{Rank function}
Let $v_1, \dots, v_n \in \mathbb{R}^d$ vectors, and 

\begin{align}
f(S) = \rank (S) = \dim \text{span} \qty\big( \qty{ v_i | i \in S} )
\end{align}


\section{Submodular optimization}
Given world $N$, submodular function $f: 2^N \to \mathbb{R}^+$, and a family of feasible solutions $\mathcal{I} \subseteq 2^N$
\begin{align}
\max \:& f(S)\\
\st & S\in \mathcal{I}
\end{align}

\paragraph{Note} Most of submodular functions (except for logarithm of determinant of submatrix) are nonnegative. We use the condition to have properly defined multiplicative approximation.
\paragraph{Note} How $f$ is given in input? Obviously, not as a list of values, since it's exponential in $\abs{N}$. Thus we represent $f$ with black box, and same applies for constraints. Usually, constraints are simple. 

\subsection{Examples of submodular optimization problems}
\paragraph{Example}
$f$ is submodular and there are no constraints. It generalizes \textsc{max-cut}, \textsc{max-dicut}
\paragraph{Example}
$f$ is submodular and there is size constraint:

\begin{align}
\max \: & f(S)\\
\st & \abs{S} \leq k
\end{align}
. It generalizes \textsc{max-k-cover}.

\paragraph{Submodular welfare}
Given $k$ players and $n$ goods, each player has submodular, monotone, non-negative value function $T_i$ over subsets of $N$. The goal is to maximize total value by partitioning the goods.
\section{Maximization of the submodular function with cardinality constraints}
\begin{align}
\max \: & f(S)\\
\st & \abs{S} \leq k
\end{align}

\paragraph{Greedy algorithm}
If $f$ is monotonic, greedy algorithm is an optimal approximating algorithm.
\begin{algorithm}
	\caption{Nemhauser-Wolsey-Fisher}\label{greedy}
	\begin{algorithmic}[1]
		\Procedure{Greedy}{$N$}
		\State $A \gets \emptyset$
		\For{$i=1$ to $k$}
		\State Let $u_i\in N$ maximize $f(A_{i-1} \cup \qty{u_i}) - f(A_{i-1})$
		\State $ A_i \gets A_{i-1} \cup \qty{u_i}$
		\EndFor
		\State \textbf{return} $A_k$
		\EndProcedure
	\end{algorithmic}
\end{algorithm}
\begin{lemma}
	For submodular $f: 2^N \to \mathbb{R}_+$,
	\begin{align}
	f(A\cup B) - f(A) \leq \sum_{b_i \in B} f(A\cup \qty{b_i}) - f(A)
	\end{align}
	\begin{proof}
		\begin{align}
		f(A\cup B) - f(A) = \sum_i f(A \cup \qty{b_1, \dots b_{i-1}} \cup \qty{b_i} ) - f(A \cup \qty{b_1, \dots b_{i-1}})\leq \sum_i f(A \cup \qty{b_i}) - f(A)
		\end{align}
	\end{proof}
\end{lemma}
\begin{prop}[\citet{nemhauser1978analysis}] \label{greedy_proof}
	\cref{greedy} is $1-\frac{1}{e}$ optimal.
	\begin{proof}
		For optimal set $S^*$
		\begin{align}
		f(A_{i-1} \cup \qty{u_i}) - f(A_{i-1})&\geq \max\limits_{u\in S^*} \qty{f(A_{i-1} \cup \qty{u}) - f(A_{i-1})} \geq \frac{1}{k} \sum_{u\in S^*} \qty[f(A_{i-1} \cup \qty{u}) - f(A_{i-1})] \geq\\&\geq \frac{1}{k} \qty\bigg(f(A_{i-1} \cup S^*) - f(A_{i-1})) \geq \frac{1}{k} \qty\bigg[f(S^*) - f(A_{i-1})]
		\end{align}
		
		We got a recursion equation:
		\begin{align}
		f(A_{i-1} \cup \qty{u_i}) - f(A_{i-1})\geq \frac{1}{k} \qty\bigg[f(S^*) - f(A_{i-1})]
		\end{align}
		We can solve the recursion and acquire
		\begin{align}
		f(A_{k}) \geq \qty(1-\qty(1-\frac{1}{k})^k)f(S^*) + \qty(1-\frac{1}{k})^k f(A_0) \geq \qty(1-\frac{1}{e}) f(S^*) 
		\end{align}
		
	\end{proof}
\end{prop}

\begin{theorem}[\citet{nemhauser1978best}] 
	For all constant $\epsilon>0$ each algorithm acquiring $1-\frac{1}{e} + \epsilon$ requires exponential number of requests to value oracle.
\end{theorem}
\begin{theorem}[\citet{feige1998threshold}]
For \textsc{max-k-cover} all constant $\epsilon>0$ each algorithm acquiring $1-\frac{1}{e} + \epsilon$ requires exponential number of requests to value oracle unless $P=NP$.
\end{theorem}
\paragraph{Note} Runtime of algorithm is $\order{nk}$. It is possible to acquire $\order{n\lg(\frac{1}{\epsilon})}$ runtime and $1-\frac{1}{e}-\epsilon$ optimality by looking on some subset of $N$ at each step instead of the whole set. 
\subsection{Non-monotonic functions}
What happens if $f$ is not monotonic? First of all, does greed algorithm work? Not only it is not optimal approximation, it can be as bad as $\frac{2}{N}$. However, it can be fixed. The idea is to randomize algorithm to prevent it from ``bad'' choices.


\paragraph{Randomized greedy algorithm}
\begin{algorithm}
	\caption{}\label{rand_greedy}
	\begin{algorithmic}[1]
		\Procedure{Randomized Greedy}{$N$}
		\State $A \gets \emptyset$
		\For{$i=1$ to $k$}
		\State $ M_i \gets \arg\max\limits_{B\subseteq N \: : \: \abs{B} \leq k} \sum_{u\in B} f\qty(A_{i-1} \cup \qty{u} ) - f\qty(A_{i-1})$
		\State $ A_i \gets \begin{cases}
		A_{i-1} \cup \qty{u} & \forall u \in M_i \text{ with } P=\frac{1}{k}\\
		A_{i-1} & \text{ with } P=1-\frac{\abs{M_i}}{k}
		\end{cases} $
		\EndFor
		\State \textbf{return} $A_k$
		\EndProcedure
	\end{algorithmic}
\end{algorithm}

\begin{prop}
	Given set $S$ and set $A$ such that each element is in $A$ with probability less than $p$
	\begin{align}
	\mathbb{E} \qty[f(S\cup A)] \geq (1-p)f(S)
	\end{align}
\end{prop}
\begin{theorem}[\citet{buchbinder2014submodular}]
	In monotonic case, \cref{rand_greedy} is $1-\frac{1}{e}$ optimal in expectation.
	\begin{proof}
		Take a look at $i^{th}$ iteration and condition on previous iterations, denote a chosen element from $M_i$ as $u_i$:
		\begin{align}
		\mathbb{E} \qty[f(A_{i-1} \cup \qty{u_i}) - f(A_{i-1}) | A_{i-1} ] &= \frac{1}{k} \sum_{u_i \in M_i} f(A_{i-1} \cup \qty{u_i}) - f(A_{i-1}) \geq \frac{1}{k} \sum_{u_i \in S^*} f(A_{i-1} \cup \qty{u_i}) - f(A_{i-1}) \geq\\&\geq \frac{1}{k} ( f(S^*) - f(A_{i-1}))
		\end{align}
		
		If the inequality is right for any $A_{i-1}$ it is right, from tower property, in expectation over $A_{i-1}$:
		\begin{align}
		\mathbb{E} \qty[f(A_{i-1} \cup \qty{u_i}) - f(A_{i-1}) ] \geq 	\frac{1}{k} ( f(S^*) - \mathbb{E} \qty[ f(A_{i-1})] )
		\end{align}
		
		And thus we can once again solve the recurrence and acquire same result as in \cref{greedy_proof}.
	\end{proof}
\end{theorem}

\begin{lemma} \label{lem:exp_emptyset}
	Given  set $B\subseteq N$ such that
	\begin{align}
	\forall u \in  N \quad P(u\in B) \leq p
	\end{align}
	
	then
	\begin{align}
	\mathbb{E} \qty[f(B)] \geq (1-p)f(\emptyset)
	\end{align}
	\begin{proof}
		WLoG $p(u_1\in B) \geq p(u_2\in B) \geq \dots \geq p(u_n\in B)$. Denote 
		\begin{align}
		X_i = \mathds{1}_{u_i \in B}
		N_i = \bigcup_{j=1}^i u_j
		\end{align}
		We can then rewrite
		\begin{align}
		f(B) &= f(N_0) + \sum_{i=1}^n X_i\qty\bigg(f(B\cap N_{i}) - f(B\cap N_{i-1}))
		\end{align}
		\begin{align}
		\mathbb{E} \qty[f(B)] &= f(N_0) +\sum_{i=1}^n \mathbb{E} \qty[ X_i\qty\bigg(f(B\cap N_{i}) - f(B\cap N_{i-1}))]\geq\\&\geq f(N_0) +\sum_{i=1}^n \qty\bigg(f(N_{i}) - f(N_{i-1}))\mathbb{E} \qty[ X_i] =f(N_0) +\sum_{i=1}^n \qty\bigg(f(N_{i}) - f(N_{i-1}))p_i =\\&= f(N_0) (1-p_1) + \sum_{i=1}^n f(N_i)\underbrace{ (p_i-p_{i+1})}_{\leq 0} \geq f(N_0) (1-p_1)  \geq f(\emptyset) (1-p)
		\end{align}
		
	\end{proof}
\end{lemma}
\begin{lemma} \label{lem:exp}
	Given set $A\subseteq N$ and set $B\subseteq N$ such that
	\begin{align}
	\forall u \in  N \quad P(u\in B) \leq p
	\end{align}
	
	\begin{align}
	\mathbb{E} \qty[f(A\cup B)] \geq (1-p)f(A)
	\end{align}
	
	\begin{proof}
		Define
		\begin{align}
		g_A(S) = f(A\cup S) 
		\end{align}
		Obviously, $g_A$ is also submodular (from diminishing returns). Then, from \vref{lem:exp_emptyset}
		\begin{align}
		\mathbb{E} \qty[f(A\cup B)] = \mathbb{E} \qty[g(B)]\geq (1-p)g(\emptyset) = (1-p)f(A)
		\end{align}
	\end{proof}
\end{lemma}

\begin{theorem}[\citet{buchbinder2014submodular}]
	In non-monotonic case, \vref{rand_greedy} is $\frac{1}{e}$ optimal in expectation.
	\begin{proof}
		Similarly to monotonic case, take a look at $i^{th}$ iteration and condition on previous iterations, denote a chosen element from $M_i$ as $u_i$:
		\begin{align}
		\mathbb{E} \qty\bigg[f(A_{i-1} \cup \qty{u_i}) - f(A_{i-1}) | A_{i-1} ] &= \frac{1}{k} \sum_{u_i \in M_i} f(A_{i-1} \cup \qty{u_i}) - f(A_{i-1}) \geq \frac{1}{k} \qty(f(A_{i-1} \cup S^*) - f(A_{i-1}) )
		\end{align}
		
		
		Since
		\begin{align}
		P(u\in A_{i-1}) \leq 1 - \qty(1-\frac{1}{k})^{i-1}
		\end{align}
		from \vref{lem:exp}
		\begin{align}
		\mathbb{E}\qty[f(A_{i-1} \cup S^*)] \geq \qty(1-\frac{1}{k})^{i-1}f(S^*)
		\end{align}
		Thus, taking expectation		
		\begin{align}
		\mathbb{E} \qty\bigg[f(A_{i-1} \cup \qty{u_i}) - f(A_{i-1}) ] &\geq \frac{1}{k} \qty(f(A_{i-1} \cup S^*) - f(A_{i-1}) ) \geq \frac{1}{k}  \qty[\qty(1-\frac{1}{k})^{i-1}f(S^*) - \mathbb{E} \qty\bigg[f(A_{i-1})] ]
		\end{align}
			
		\begin{align}
		\mathbb{E} \qty\bigg[f(A_{i} )] &\geq \frac{1}{k} \qty(f(A_{i-1} \cup S^*) - f(A_{i-1}) ) \geq \frac{1}{k}  \qty[\qty(1-\frac{1}{k})^{i-1}f(S^*) - \mathbb{E} \qty\bigg[f(A_{i-1})] ]
		\end{align}
		Solving the recurrence we get
		\begin{align}
		\mathbb{E} \qty[f(A_i)] \geq \frac{i}{k} \qty(1-\frac{1}{k})^{k-1} f(S^*) \geq \frac{1}{e}f(S^*) 
		\end{align}
		i.e.,
		\begin{align}
		\mathbb{E} \qty[f(A_k)] \geq  \qty(1-\frac{1}{k})^{k-1} f(S^*) \geq \frac{1}{e}f(S^*) 
		\end{align}
		
		
	\end{proof}
\end{theorem}

\paragraph{Note} \vref{rand_greedy} is not optimal. In addition, the upper bound of the best approximation is $~0.49$.
\paragraph{Runtime} Runtime of \vref{rand_greedy} is $\order{nk}$.

\section{Maximization of the submodular function without constraints}
\begin{align}
\max \: & f(S)
\end{align}
\paragraph{Examples}
\begin{itemize}
\item \textsc{max-cut}
\item \textsc{max-directed-cut}
\item Max Facility Location
\item \textsc{max-sat} (with all literals in a clause having same sign).
\end{itemize}

\begin{prop}[\cite{feige2011maximizing}]
	Algorithm which choose random solution as following: $u\in S$ with probability $\frac{1}{2}$ independently, is $\frac{1}{4}$ approximation in expectation:
	\begin{align}
	\mathbb{E}[f(S)] \geq \frac{1}{4} f(S^*)
	\end{align}
\end{prop}
\begin{prop}[\cite{feige2011maximizing}]
If $f$ is symmetric, the same algorithm is $\frac{1}{2}$ approximation in expectation:
\begin{align}
\mathbb{E}[f(S)] \geq \frac{1}{2} f(S^*)
\end{align}
\end{prop}
\begin{prop}[\cite{feige2011maximizing}]
For any constant $\epsilon>0$ it is impossible to acquire $\qty(\frac{1}{2}+\epsilon)$ approximation in polynomial time, even in  symmetric case.
\end{prop}

Note that for $\bar{f}(S) = f(\bar{S})$, we can use the same oracle. So a ''conjugate`` algorithm would be start from $N$ and drop elements from it.


\begin{algorithm}
	\caption{}\label{alg:double_greedy}
	\begin{algorithmic}[1]
		\Procedure{Double Greedy}{$N$}
		\State $X \gets \emptyset$, $Y \gets N$
		\For{$i=1$ to $n$}
		\State $a_i = f(X_{i-1} \cup \qty{u_i})-f(X_i)$
		\State $b_i = f(Y_{i-1} \setminus \qty{u_i})-f(Y_i)$
		\If{$a_i > b_i$}
		\State $X_i \gets X_{i-1} \cup \qty{u_i}$
		\State $Y_i \gets Y_{i-1}$
		\Else
		\State $X_i \gets X_{i-1}$
		\State $Y_i \gets Y_{i-1} \setminus \qty{u_i}$
		\EndIf
		\EndFor
		\State \textbf{return} $X_N$
		\EndProcedure
	\end{algorithmic}
\end{algorithm}


\begin{algorithm}
	\caption{}\label{alg:rand_double_greedy}
	\begin{algorithmic}[1]
		\Procedure{Randomized Double Greedy}{$N$}
		\State $X \gets \emptyset$, $Y \gets N$
		\For{$i=1$ to $n$}
		\State $a_i = \max \qty{0,f(X_{i-1} \cup \qty{u_i})-f(X_i)}$
		\State $b_i = \max \qty{0,f(Y_{i-1} \setminus \qty{u_i})-f(Y_i)}$
		\State $(X_i,Y_i) \gets\begin{cases}
		\qty(X_{i-1} \cup \qty{u_i}, Y_i) & \text{with } P= \frac{a_i}{a_i+b_i} \\
		\qty(X_{i-1} ,Y_{i-1} \setminus \qty{u_i}) & \text{with } P= \frac{b_i}{a_i+b_i} 
		\end{cases}$
		
		\EndFor
		\State \textbf{return} $X_N$
		\EndProcedure
	\end{algorithmic}
\end{algorithm}
\begin{prop}
	It's impossible that both $f(X_{i-1} \cup \qty{u_i})-f(X_{i-1}) <0$ and  $f(Y_{i-1} \setminus \qty{u_i}) - f(Y_i)<0$.
	\begin{proof}
		From diminishing returns:
		\begin{align}
		f(X_{i-1} \cup \qty{u_i})-f(X_{i-1}) \geq f(Y_i) - f(Y_{i-1} \setminus \qty{u_i})  \\
		f(X_{i-1} \cup \qty{u_i})-f(X_{i-1}) + f(Y_i) - f(Y_{i-1} \setminus \qty{u_i}) \geq 0
		\end{align}
		Thus at least one of  $f(X_{i-1} \cup \qty{u_i})-f(X_{i-1}) $ and  $f(Y_{i-1} \setminus \qty{u_i}) - f(Y_i)$ is greater than $0$.
	\end{proof}
\end{prop}


\begin{lemma}
	Let $S^*$ be an optimal solution and 
	\begin{align}
	S^*_i = S^* \cup X_i \cap Y_i
	\end{align}
	i.e., optimal solution to which we add everything \cref{alg:double_greedy} added and drop everything it dropped.
	
	For all $i$:
	\begin{align}
	f(S^*_{i-1}) - f(S^*_i) \leq f(X_i) - f(X_{i-1})+f(Y_i) - f(Y_{i-1}) 
	\end{align}
\end{lemma}

\begin{lemma}
	Let $S^*$ be an optimal solution and 
	\begin{align}
	S^*_i = S^* \cup X_i \cap Y_i
	\end{align}
	i.e., optimal solution to which we add everything \cref{alg:rand_double_greedy} added and drop everything it dropped.
	
	For all $i$:
	\begin{align}
	\mathbb{E} \qty\bigg[f(S^*_{i-1}) - f(S^*_i)] \leq \frac{1}{2} \mathbb{E} \qty\bigg[f(X_i) - f(X_{i-1})+f(Y_i) - f(Y_{i-1}) ]
	\end{align}
	
	\begin{proof}
		Take a look at $i^{th}$ iteration and condition on previous iterations:
		\begin{align}
		\mathbb{E} \qty\bigg[ f(X_i) - f(X_{i-1}) + f(Y_i) - f(Y_{i-1}) \bigg\vert X_{i-1}, Y_{i-1}] =\\= \frac{a_i}{a_i+b_i} \underbrace{\qty(f(X_{i-1} \cup \qty{u_i})-f(X_{i-1}))}_{=a_i  \text{ if } a_i\neq 0} + \frac{b_i}{a_i+b_i} \underbrace{\qty(f(Y_{i-1} \cup \qty{u_i})-f(Y_{i-1}))}_{=b_i  \text{ if } b_i\neq 0} = \frac{a_i^2+b_i^2}{a_i+b_i}
		\end{align}
		
		Now divide into two cases: $u_i \in S^*$ and $u_i \notin S^*$.
		
		\begin{itemize}
			\item If $u_i \notin S^*$, in particular, $u_i \notin S^*_{i-1}$:
			\begin{align}
			\mathbb{E} \qty[f(S^*_{i-1})-f(S^*_{i}) ] &= \frac{a_i}{a_{i} + b_{i}} \qty(f(S^*_{i-1}) -f(S^*_{i-1}\cup \qty{u_i})) \stackrel{S^*_{i-1} \subseteq Y_{i-1}\setminus \qty{u_i}}{\leq }\\&\leq \frac{a_i}{a_{i} + b_{i}} \qty(f(Y^*_{i-1}\setminus \qty{u_i}) -f(Y^*_{i-1})) \leq \frac{a_i b_i}{a_i+b_i}
			\end{align}
			\item If $u_i \in S^*$, in particular, $u_i \in S^*_{i-1}$:
			\begin{align}
			\mathbb{E} \qty[f(S^*_{i-1})-f(S^*_{i}) ] &= \frac{b_i}{a_{i} + b_{i}} \qty(f(S^*_{i-1}) -f(S^*_{i-1}\setminus \qty{u_i})) \stackrel{X_{i-1} \subseteq S^*_{i-1} \setminus \qty{u_i}}{\leq }\\&\leq \frac{b_i}{a_{i} + b_{i}} \qty(f(X^*_{i-1}\cup \qty{u_i}) -f(X^*_{i-1})) \leq \frac{a_i b_i}{a_i+b_i}
			\end{align}
		\end{itemize}
		And since $a_i^2-2a_ib_i+b_i^2 = (a_i-b_i)^2 \geq 0$ (and by tower property), we get the required.
		
	\end{proof}
\end{lemma}


\begin{theorem}[\cite{buchbinder2015tight}]
	\vref{alg:rand_double_greedy} is $\frac{1}{2}$ approximation in expectation.
	\begin{proof}
		Denote 
		\begin{align}
			S_{alg} = S^*_n=X_n=Y_n
		\end{align}
		Then
		\begin{align}
		\mathbb{E} \qty\bigg[f(S^*_{0}) - f(S^*_n)] &\leq \frac{1}{2} \mathbb{E} \qty\bigg[f(X_n) - f(X_{0})+f(Y_n) - f(Y_{0}) ]\\
		\mathbb{E} \qty\bigg[f(S^*) - f(S_{alg})] &\leq \frac{1}{2} \mathbb{E} \qty\bigg[2S_{alg} - f(X_{0}) - f(Y_{0}) ]\stackrel{f(S)\geq 0}{\leq} \mathbb{E} \qty\bigg[S_{alg} ]
		\end{align}
		Thus
		\begin{align}
		\mathbb{E} \qty[S_{alg} ] \geq\frac{1}{2}\mathbb{E} \qty[f(S^*)] 
		\end{align}
		
	\end{proof}

\begin{coll}
	\vref{alg:double_greedy} is $\frac{1}{3}$ approximation.
\end{coll}
\end{theorem}


\paragraph{Note} \vref{alg:double_greedy,alg:rand_double_greedy} run in $\order{N}$ time.


\section{Knapsack constraints}
Let each element of set have price $c_i$ and budget $B$, then
\begin{align}
\max \: & f(S)\\
\st & \sum_{i \in S} c_i \leq B
\end{align}


\begin{algorithm}
	\caption{}\label{alg:knapsack_greedy}
	\begin{algorithmic}[1]
		\Procedure{Density Greedy}{$N$}
		\State $S \gets \emptyset$
		\While{$N\neq \emptyset$}
		\State $x^* \gets \arg \max \qty{ \frac{f(S\cup \qty{x})-f(S)}{c_i}}$
		\If{$c(S) + c_{x^*} \leq B$}
		\State $S\gets S \cup \qty{x^*}$
		\EndIf
		\State $N \gets N\setminus \qty{x^*}$
		\EndWhile
		\State \textbf{return} $S$
		\EndProcedure
	\end{algorithmic}
\end{algorithm}

Note that this is generalization of cardinality constraint.


\begin{algorithm}
	\caption{}\label{alg:optimized_knapsack}
	\begin{algorithmic}[1]
		\Procedure{Optimized Density Greedy}{$N$}
		\State $S_1 \gets$ output of \vref{alg:knapsack_greedy}
		\State $S_2 \gets \qty{\arg\max\limits_{\substack{i\in N\\c_i \leq B}} f(i)}$
		\State \textbf{return} $\arg\max\limits_{S\in \qty{S_1,S_2}} f(S)$
		\EndProcedure
	\end{algorithmic}
\end{algorithm}


\begin{prop}[\cite{khuller1999budgeted}]
	\vref{alg:optimized_knapsack} is $\frac{1}{2}\qty(1-\frac{1}{e})$-optimal.
\end{prop}
\begin{prop}
	\vref{alg:optimized_knapsack} is $\qty(1-\frac{1}{\sqrt{e}})$-optimal.
\end{prop}

\begin{theorem}[\cite{khuller1999budgeted,sviridenko2004note}]
	% https://thibaut.horel.org/submodularity/notes/02-12.pdf
	If a set of $l$ most dense items in optimal solution $S^*$, it is possible to get good approximation to the optimal solution.
	
	Enumerating all sets of up to 3 most dense items in optimal solution $S^*$, we can acquire $1-\frac{1}{e}$-approximation of optimal solution. Since cardinality constraint is a particular case of knapsack constraint, this is best polynomial approximation.
\end{theorem}
\section{Introduction to matroids}
Matroid is a basic concept in combinatorial optimization. It was first defined by \citet{whitney1935abstract}. 
\begin{definition}[matroid]
	Matroid $\mathcal{M}$ is a pair $(E,\mathcal{I})$. $E$ is is a finite set (called the ground set) and $\mathcal{I}\neq \emptyset$ is a family of subsets of $E$ (called the independent sets) with the following properties:
	\begin{enumerate}
		\item If $Y\in \mathcal{I}$ then for all $X\subseteq Y$, $X\in \mathcal{I}$.
		\item If $X,Y\in \mathcal{I}$ and $\abs{Y}>\abs{X}$, then exists $e\in Y\setminus X$, $X\cup \qty{e} \int \mathcal{I}$.
	\end{enumerate}
\end{definition}

\paragraph{Notes} All maximal independent sets have same size. Those sets are called basis.


\paragraph{Examples}
\subparagraph{Uniform manifold} 
\begin{align}
	\mathcal{M}_{k} = \qty(E, \qty\bigg{X\subseteq E | \abs{X} \leq k})
\end{align}
\subparagraph{Linear manifold} 
Let $A\in \mathbb{R}^{m\times n}$ be a matrix. Let $E$ be a set of columns of $A$. The set $X\subseteq E$ is independent if its elements are independent. Alternatively, for sub-matrix $A_X$ consisting of columns of $A$:
\begin{align}
\mathcal{I} = \qty{ X\subset E | \rank(A_x) = \abs{X}}
\end{align}


\subparagraph{Graphic matroids} 
Let $G=\qty(V_G,E_G)$ be a graph, $E=E_G$ and
\begin{align}
\mathcal{I} = \qty{X\subseteq E_G | X \text{ is forest}}
\end{align}
\begin{prop}
	$M=(E_G,\mathcal{I})$ is matroid.
\end{prop}
The basis is then spanning trees (or forests if graph is not connected).

\subparagraph{Partition matroid}
For a set $E$ let $E_1,\dots E_k$ be some partition of $E$. Then
\begin{align}
\mathcal{I}= \qty\big{ X\subseteq E \big\vert \forall i=1..k \: \abs{X}\cup E_i  \leq 1  }
\end{align}
\begin{prop}
	$M=(E,\mathcal{I})$ is matroid.
\end{prop}
Note that partition matroid encodes constraints of submodular welfare problem.

Constraint of matching in the bipartite graph can be defined as intersection of two partition matroids.

\begin{definition}[Circuit]
	Circuit in matroid $M=(E,\mathcal{I})$ is a dependent set $X$ ($X\notin \mathcal{I}$) and for all $x\in X$, $X\setminus \qty{x} \in \mathcal{I}$.
\end{definition}
\begin{definition}[Rank function]
For matroid $M=(E,\mathcal{I})$ rank function $r: 2^{\mathbb{E}} \to \mathbb{N}$ is defined as
\begin{align}
r(x) = \max \qty\big{\abs{Y} \big| Y\subset{X}, Y\in \mathcal{I}}
\end{align}
\end{definition}
\begin{definition}[Rank of matroid]
For matroid $M=(E,\mathcal{I})$ rank of matroid is $\rank(E)$.
\end{definition}
\begin{prop}
	Rank of matroid is submodular function.
\end{prop}


\begin{algorithm}
	\caption{}\label{alg:matroid_basis}
	\begin{algorithmic}[1]
		\Procedure{Greedy}{$E$, $I$}
		\State $S \gets \emptyset$
		\For {$e\in E$}
		\If {$S\cup \qty{x} \in \mathcal{I}$}
		\State $S\gets S\cup \qty{x}$
		\EndIf
		\EndFor
		\State \textbf{return} $S$
		\EndProcedure
	\end{algorithmic}
\end{algorithm}

\begin{prop}
	\vref{alg:matroid_basis} returns basis of $E$.
	\begin{proof}
		Assume $S$ is not a basis and let $B$ be a basis. Exists $x\in B\setminus S$ such that $S\cup \qty{x}$ is independent. However, since we have not added $x$ to $S$, it got to be dependent with $S$.
	\end{proof}
\end{prop}

\paragraph{Question} Given matroid over $E$ (via independence oracle), let weight function $w: E\to \mathbb{R}$ and weight of set be $w(X) = \sum_{x\in X} w(x)$. We want to find independent set (pr basis) of maximal weight.


\begin{algorithm}
	\caption{}\label{alg:matroid_weighted_basis}
	\begin{algorithmic}[1]
		\Procedure{Greedy}{$E$, $I$}
		\State $S \gets \emptyset$
		\For {$e\in E$ from heaviest to lightest}
		\If {$S\cup \qty{x} \in \mathcal{I}$}
		\State $S\gets S\cup \qty{x}$
		\EndIf
		\EndFor
		\State \textbf{return} $S$
		\EndProcedure
	\end{algorithmic}
\end{algorithm}
\begin{prop}
	\vref{alg:matroid_weighted_basis} solves the problem of maximal weight basis.
	\begin{proof}
		We know that for $k = \rank(M)$, the size of the output of algorithm is $k$ and so is size of optimal solution $S^*$.
		Lets assume $S$ is not optimal, thus exists $i$ such that $w(e_i^*) > w(e_i)$.
		
		At iteration at which we added $e_i$ to $S$. At this iteration $\abs{S}=i-1$. take a look at first $i$ elements of $S^*$: this is independent set, and from definition of matroid, exists $e'$ such that $S$ is independent with $e'$. However, $w(e') \geq w(e^*-i) > w(e_i)$, thus we should have added it beforehand.
	\end{proof}
\end{prop}

\section{Continuous extensions of submodular functions} 
\begin{definition}[Continuous extensions of function]
	Denote by $\mathds{1}_S \in \qty{0,1}^N$ indicator of $S$. For $f: 2^N \to \mathbb{R}$ extension of $f$ is $F: [0,1]^N \to \mathbb{E}$ such that for all $S\in N$ $F(\mathds{1}_S) = f(S)$.
\end{definition}

There exist many extensions of submodular functions. In particular, there exist convex and concave extensions of submodular functions. 
\paragraph{Main idea}
For point $x\in [0,1]^N$ define distribution on subsets of $N$, $D_x$ such that for $R\sim D_x$:
\begin{align}
	P(i \in R) = x_i
\end{align}
Then we define 
\begin{align}
	F(x) = \mathbb{E}_{R\sim D_x} [f(R)]
\end{align}

\paragraph{Example}
For all $x$ choose $D_x$ such that $F(x)$ is maximized:
\begin{align}
f^+(x)  = \max\limits_{D_x}  \mathbb{E}_{R\sim D_x} [f(R)]
\end{align}
Similarly, we can choose $D_x$ such that $F(x)$ is minimized:
\begin{align}
f^-(x)  = \min\limits_{D_x}  \mathbb{E}_{R\sim D_x} [f(R)]
\end{align}

\begin{prop}
	$f^+$ is concave and $f^-$ is convex.
	\begin{proof}
		Let $x,y\in [0,1]^N$ lets show that for
		\begin{align}
			z = \lambda x + (1-\lambda)y
		\end{align}
		concave property is fulfilled:
		\begin{align}
		f^=(z) \geq \lambda f^+(x) +  (1-\lambda)f^+(y) 
		\end{align}
		Let $\qty{\alpha_S}_{S\subseteq N}$ be a distribution which defined $f(x)$: $P(R = S) = \alpha_S$, which fulfills:
		\begin{align}
		f^+(x) &= \mathbb{E}_{R\sim \alpha_S} [f(R)] \\
		\sum_{S: i\in S} \alpha_S &= x_i
		\end{align}
		Similarly, let $\qty{\beta_S}_{S\subseteq N}$ be a distribution defining $f(y)$.
		
		Now, take a look at linear combination of $\alpha_S$ and $\beta_S$:
		\begin{align}
		P(S) = \lambda\alpha_S + (1-\lambda)\beta_S
		\end{align}
		Note that this distribution conserves marginal values of $z$, since:
		\begin{align}
		\sum_{S: i\in S} \lambda \alpha_S + (1-\lambda) \beta_S = \lambda \sum_{S: i\in S} \alpha_S + (1-\lambda) \sum_{S: i\in S}\beta_S = \lambda x_i + (1-\lambda) y_i = z_i
		\end{align} 
		By definition,
		\begin{align}
		f^+(z) \geq  \mathbb{E}_{R\sim\lambda \alpha_S + (1-\lambda) \beta_S} [f(R)] = \sum_{S \subseteq N} P(R=S) f(S) = \sum_{S \subseteq N} \qty[\lambda \alpha_S + (1-\lambda) \beta_S]f(S) = \lambda f^+(x) + (1-\lambda) f^+(y)
		\end{align}
	\end{proof}
\end{prop}

\begin{prop}
	Evaluating concave extensions of submodular function in some point is NP-hard.
\end{prop}

\begin{definition}[Lovasz extension]
	\begin{align}
	f_L(x) = \mathbb{E}_{\theta \sim [0,1]} [f\qty\big(\qty{i: x_i \geq \theta})]
	\end{align}
\end{definition}

\begin{theorem}[Lovasz]
	$f_L(x) = f^-(x)$ iff $f$ is submodular.
	\begin{proof}
		$\Leftarrow$:
		Denote $\qty{ \alpha_S}_{S\subseteq N}$ a distribution defining $f^-(x)$, and out of those the one that maximizes $\sum_{S \subseteq N} \alpha_S \abs{S}^2$.
		
		We'll show that for such $\alpha$, the sets for which $\alpha_S >0$ are chain (i.e., a set of sets such that for $A$, $B$ in the set either $A\subseteq B$ or $B\subseteq A$).
		
		Note that there is unique distribution that conserves marginal values and its support is chain: the Lovasz distribution. 
		
		Lets show how we can ''fix`` the distribution $\alpha_S$ which support is not a chain (uncrossing). Suppose there are $A,B$ such that $\alpha_A\geq \alpha_B >0$ and $A\not\subseteq B$ and  $B\not\subseteq A$.
		For that, lets reduce the probability of $A$ and $B$ by $\alpha_B$, and increase probability of $A\cap B$ and $A\cup B$ by $\alpha_B$.
		
		Does the new distribution conserve marginal values? For all of cases $x\in A\cap B$, $x\in A\setminus B$ and $x\in B\setminus A$, the probability did not change.
		
		What happened to $\mathbb{E} [f(R)]$? From submodularity,
		\begin{align}
		f(A) + f(B) \geq f(A\cap B) + f(A\cup B)
		\end{align}
		and since we removed LHS and added RHS multiplied same constant, the expectation can not grow.
		
		What happens to $\sum_{S \subseteq N} \alpha_S \abs{S}^2$?
		\begin{align}
		\abs{A\cup B}^2 + \abs{A\cap B}^2 &= \qty(\abs{A} + \abs{B\setminus A})^2 + \qty(\abs{B} - \abs{B\setminus A})^2 = \abs{A}^2+\abs{B}^2 + 2\abs{B\setminus A} \qty(\abs{A} - \abs{B} + \abs{B\setminus A}) =\\&= \abs{A}^2+\abs{B}^2 + 2\abs{B\setminus A}\underbrace{ \qty(\abs{A\cup B} - \abs{B})}_{>0} >  \abs{A}^2+\abs{B}^2
		\end{align}
		
		Thus, if we choose the set which maximizes $\sum_{S \subseteq N} \alpha_S \abs{S}^2$, there are no two sets $A,B$ such that $A\not\subseteq B$ and $B \not\subseteq A$, i.e., support is the chain.
		
		
		
		$\Rightarrow$:
	\end{proof}
\end{theorem}


\begin{definition}[Multilinear extension]
	\begin{align}
	F(x) = \sum_{S \subseteq N} f(S) \prod_{i\in S} x_i \prod_{i \notin S} (1-x_i)
	\end{align}
	i.e., each element is chosen independently.
\end{definition}


\begin{prop}
	Let $f$ be monotonous function. Then for $\vb{x}\in [0,1]^N$, $\mathbb{R}^n \ni \vb{y}>0$ (coordinate-wise), and $g(t)=F(\vb{x}+t\vb{y})$, $g$ is monotonous, i.e., 
	\begin{align}
	\pdv{F}{\vb{y}}\geq 0
	\end{align}
	In other words, for $i\in N$
	\begin{align}
	\pdv{F}{x_i} \geq  0
	\end{align}
\end{prop}
\begin{prop}
	Let $f$ be submodular function. Then for $\vb{x}\in [0,1]^N$, $\mathbb{R}^n \ni \vb{y}>0$ (coordinate-wise), and $g(t)=F(\vb{x}+t\vb{y})$, $g$ is concave, i.e., 
	\begin{align}
	\pdv[2]{F}{\vb{y}}\leq 0
	\end{align}
	In other words, for $i,j\in N$
	\begin{align}
	\pdv{F}{x_i}{x_j}\leq 0
	\end{align}
\end{prop}

\section{Matroid constraints}
Let $\mathcal{M}=(E,\mathcal{I})$ be a matroid.
\begin{align}
\max \: & f(S)\\
\st & S\in \mathcal{I}
\end{align}

Let 
\begin{align}
\mathcal{P}_\mathcal{M} = \qty{z\in [0,1]^N \Big\vert \forall S\subseteq N \: \sum_{i \in S}z_i\leq \rank(S)}
\end{align}

\begin{algorithm}
	\caption{}\label{alg:cont_greedy}
	\begin{algorithmic}[1]
		\Procedure{Continuous Greedy}{$N$}
		\State $\vb{y}(0) \gets \vb{0}$
		\For{$t'\in (0,1)$}
		\State $\vb{x}(t') \gets \arg \max\limits_{\vb{x} \in \mathcal{P}_\mathcal{M}} \qty{\vb{x}\vdot \grad{F\qty(\vb{y}(t'))}} $
		\State $\pdv{\vb{y}}{t}\qty(t') \gets \vb{x}(t') $
		\EndFor
		\State \textbf{return} $\vb{y}(1)$
		\EndProcedure
	\end{algorithmic}
\end{algorithm}

\begin{lemma}
	For $\vb{x}\in [0,1]^N$
	\begin{align}
	\sum_{i \in S} F\qty(\max\qty{\vb{x}, \mathds{1}_i})-F(\vb{x}) \geq F\qty(\max\qty{\vb{x}, \mathds{1}_{S}})-F(\vb{x})
	\end{align}
	\begin{proof}
		Denote by $D_{\vb{x}}$ random distribution of taking each element independently with probability $x_i$, i.e.,
		\begin{align}
		F(\vb{x}) = \mathbb{E}_{R\sim D_{\vb{x}}} [f(R)]
		\end{align}
		Then
		\begin{align}
		\sum_{i \in S} F\qty(\max\qty{\vb{x}, \mathds{1}_i})-F(\vb{x}) &= \sum_{i \in S} \mathbb{E}_{R\sim D_{\max\qty{\vb{x}, \mathds{1}_i}}} [f(R)] - \mathbb{E}_{R\sim D_{\vb{x}}} [f(R)] = \sum_{i \in S}\mathbb{E}_{R\sim D_{\vb{x}}}\qty[f(R\cup \qty{i}) - f(R)] =\\&= \mathbb{E}_{R\sim D_{\vb{x}}}\qty[\sum_{i \in S} f(R\cup \qty{i}) - f(R)] \geq \mathbb{E}_{R\sim D_{\vb{x}}}\qty[f(R\cup S) - f(R)] = F\qty(\max\qty{\vb{x}, \mathds{1}_{S}})-F(\vb{x})
		\end{align}
	\end{proof}
\end{lemma}
\begin{theorem}[\citet{calinescu2011maximizing}]
	For monotonous submodular $f$,
	\begin{align}
	F(\vb{y}(1)) \geq \qty(1-\frac{1}{\epsilon})f(S^*)
	\end{align}
	where $\vb{y}(t)$ is output of \vref{alg:cont_greedy}:
	\begin{align}
	\vb{y}(t) = \int_0^t \vb{x}(t') \dd{t'}
	\end{align}
	and $S^*$ is optimal solution of
	\begin{align}
	\max \: & f(S)\\
	\st & S\in \mathcal{I}
	\end{align}
	\begin{proof}
		Lets bound $\pdv{F}{t} (\vb{y}(t'))$:
		\begin{align}
		\pdv{F}{t} (\vb{y}(t')) &=  \grad{F} \vdot \pdv{y}{t} =  \grad{F\qty(\vb{y}(t'))} \vdot \vb{x}(t') \geq \grad{F\qty(\vb{y}(t'))} \vdot \mathds{1}_{S^*}= \sum_{i \in S^*} \qty(\grad{F\qty(\vb{y}(t'))})_i \stackrel{F \text{is multilinear}}{\geq} \sum_{i \in S^*} \frac{F\qty(\max\qty{\vb{y}, \mathds{1}_i})-F(\vb{y})}{1-\vb{y}_i} \geq\\&\geq \sum_{i \in S^*} F\qty(\max\qty{\vb{y}, \mathds{1}_i})-F(\vb{y}) \geq F\qty(\max\qty{\vb{y}, \mathds{1}_{S^*}})-F(\vb{y}) \geq f(S^*)-F(\vb{y})
		\end{align}
		where $\max$ is coordinate-wise maximum.
		We got
		\begin{align}
			F(\vb{y}(0)) &\geq 0 \\
			\pdv{F}{t} &\geq f(S^*) - F(\vb{y})
		\end{align}
		with solution $F(\vb{y}(t)) \geq \qty(1-e^{-t}) f(S^*)$, i.e., 
		\begin{align}
		F(1) \geq \qty(1-\frac{1}{\epsilon})f(S^*)
		\end{align}
	\end{proof}
	%i.e., \vref{alg:cont_greedy} is $ \qty(1-\frac{1}{\epsilon})$-optimal.
\end{theorem}
\paragraph{Note} $\grad F$ can be estimated efficiently with random sampling.

\begin{algorithm}
	\caption{}\label{alg:measured_cont_greedy}
	\begin{algorithmic}[1]
		\Procedure{Measured Continuous Greedy}{$N$}
		\State $\vb{y}(0) \gets \vb{0}$
		\For{$t'\in (0,1)$}
		\State $\vb{x}(t') \gets \arg \max\limits_{\vb{z} \in \mathcal{P}_\mathcal{M}} \qty{\sum_{i \in N} z_i\cdot \qty( F\qty(\max\qty{\vb{y}, \mathds{1}_i})-F(\vb{y}))} $
		\State $\qty(\pdv{\vb{y}}{t})_i\qty(t') \gets \vb{x}_i(t') \qty(1-\vb{y}_i(t'))$
		\EndFor
		\State \textbf{return} $\vb{y}(1)$
		\EndProcedure
	\end{algorithmic}
\end{algorithm}

\begin{lemma} \label{bound_random}
	For set $S\subseteq N$ and vector $\vb{z}$ such that $\norm{z}_\infty \leq p$
	\begin{align}
	F(\max\qty{\mathds{1}_S , \vb{z}}) \geq (1-p) f(S)
	\end{align}
	\begin{proof}
		\begin{align}
		F(\max\qty{\mathds{1}_S , \vb{z}}) \geq f_L(\max\qty{\mathds{1}_S , \vb{z}}) \geq (1-p)f(S)
		\end{align}
	\end{proof}
\end{lemma}
\begin{theorem}[\citet{feldman2011unified}]
	For submodular $f$,
	\begin{align}
	F(\vb{y}(1)) \geq \qty(1-\frac{1}{\epsilon})f(S^*)
	\end{align}
	where $\vb{y}(t)$ is output of \vref{alg:measured_cont_greedy}:
	\begin{align}
	\vb{y}(t) = \int_0^t \vb{x}(t') \dd{t'}
	\end{align}
	and $S^*$ is optimal solution of
	\begin{align}
	\max \: & f(S)\\
	\st & S\in \mathcal{I}
	\end{align}
	\begin{proof}
		Lets bound $\pdv{F}{t} (\vb{y}(t'))$:
		\begin{align}
		\pdv{F}{t} (\vb{y}(t')) &=  \grad{F} \vdot \pdv{y}{t} =  \grad{F\qty(\vb{y}(t'))} \vdot \vb{x}(t') \stackrel{F \text{is multilinear}}{=} \sum_{i \in N} \frac{F\qty(\max\qty{\vb{y}, \mathds{1}_i})-F(\vb{y})}{1-\vb{y}_i}\cdot \vb{x}(t') \cdot   \qty(1-\vb{y}_i(t')) =\\&=
		\qty\big[F\qty(\max\qty{\vb{y}, \mathds{1}_i})-F(\vb{y})]\cdot \vb{x}(t') \geq \sum_{i \in S^*} F\qty(\max\qty{\vb{y}, \mathds{1}_i})-F(\vb{y}) \geq F\qty(\max\qty{\vb{y}, \mathds{1}_{S^*}})-F(\vb{y})
		\end{align}
		where $\max$ is coordinate-wise maximum.
		
		Since
		\begin{align}
		\pdv{y_i}{t} \leq 1-y_1
		\end{align}
		we get
		\begin{align}
			y_i \leq 1-e^{-t}
		\end{align}
		From \ref{bound_random} we got
		\begin{align}
		\pdv{F}{t} (\vb{y}(t')) &\geq F\qty(\max\qty{\vb{y}, \mathds{1}_{S^*}})-F(\vb{y}) \geq e^{-t} f(S^*) - F(\vb{y})
		\end{align}
		Solving differential inequality
		\begin{align}
		F(\vb{y}(0)) &\geq 0 \\
		\pdv{F}{t} &\geq f(S^*) - F(\vb{y})
		\end{align}
		we acquire the solution $F(\vb{y}(t)) \geq te^{-t} f(S^*)$ with maximum in $t=1$:
		\begin{align}
		F(1) \geq \frac{1}{\epsilon}f(S^*)
		\end{align}
		
		Note that now $\vb{y}$ is not a linear combination of elements of polytope, but rather something which is at most linear combination.
	\end{proof}
	%i.e., \vref{alg:cont_greedy} is $ \qty(1-\frac{1}{\epsilon})$-optimal.
\end{theorem}

\paragraph{Note} It is known that $\frac{1}{e}$ is not optimal, but rather at least $0.372$ \cite{ene2016constrained}.

\paragraph{Rounding of the fractional solution}
Remember the submodular welfare problem. We define polytope
\begin{align}
\mathcal{P}_M = \qty{ \vb{z} \in [0,1]^N \big\vert \sum_{j=1}^k z_{i,j} \leq 1}
\end{align}
Then an intuitive way to round the vector in polytope is just sample from random variable $Z_i$:
\begin{align}
P(Z_i= j) = z_{i,j}
\end{align}
with $Z_i$ independent. Denote by $S$ the set acquired by this rounding.
\begin{prop}
	\begin{align}
	\mathbb{E}\qty[f(S)] \geq F(\vb{z})
	\end{align}
\end{prop}

\paragraph{Note} For submodular welfare with $k$ players we can get $1-\qty(1-\frac{1}{k})^k$ approximation, by running \vref{alg:measured_cont_greedy} and stopping after $t=1$.


\paragraph{Pipage rounding}
We choose two components of vector $\vb{z}$, $z_i$ and $z_j$ and move mass from $z_i$ to $z_j$ such that
\begin{itemize}
	\item At least one of  $z_i$ and $z_j$ becomes integer
	\item Total value isn't reduced (due to concavity in two variables)
	\item New vector is still in $\mathcal{P}_M$. 
\end{itemize}
\paragraph{Swap rounding}
Decompose $\vb{z}$ as a convex combination over basis. From two vectors in combination we create a new one: there exists $x\in B_i \setminus B_j$ and $y\in B_j \setminus B_i$ such that $B_i \setminus \qty{x} \cup \qty{y}$ and $B_j \setminus \qty{y} \cup \qty{x}$ is a basis.


\bibliographystyle{plainnat}
\bibliography{236621_-_Algorithms_for_Submodular_Optimization}
\end{document}
