\begin{lemma} \label{lem:exp_emptyset}
	Given  set $B\subseteq N$ such that
	\begin{align}
	\forall u \in  N \quad P(u\in B) \leqslant p
	\end{align}
	
	then
	\begin{align}
	\mathbb{E} \qty[f(B)] \geqslant (1-p)f(\emptyset)
	\end{align}
	\begin{proof}
		WLoG $p(u_1\in B) \geqslant p(u_2\in B) \geqslant \dots \geqslant p(u_n\in B)$. Denote 
		\begin{align}
		X_i = \mathds{1}_{u_i \in B}
		N_i = \bigcup_{j=1}^i u_j
		\end{align}
		We can then rewrite
		\begin{align}
		f(B) &= f(N_0) + \sum_{i=1}^n X_i\qty\bigg(f(B\cap N_{i}) - f(B\cap N_{i-1}))
		\end{align}
		Then
		\begin{align}
		\mathbb{E} \qty[f(B)] &= f(N_0) +\sum_{i=1}^n \mathbb{E} \qty[ X_i\qty\bigg(f(B\cap N_{i}) - f(B\cap N_{i-1}))]\geqslant\\&\geqslant f(N_0) +\sum_{i=1}^n \qty\bigg(f(N_{i}) - f(N_{i-1}))\mathbb{E} \qty[ X_i] =f(N_0) +\sum_{i=1}^n \qty\bigg(f(N_{i}) - f(N_{i-1}))p_i =\\&= f(N_0) (1-p_1) + \sum_{i=1}^n f(N_i)\underbrace{ (p_i-p_{i+1})}_{\leqslant 0} \geqslant f(N_0) (1-p_1)  \geqslant f(\emptyset) (1-p)
		\end{align}
		
	\end{proof}
\end{lemma}
\begin{lemma} \label{lem:exp}
	Given set $A\subseteq N$ and set $B\subseteq N$ such that
	\begin{align}
	\forall u \in  N \quad P(u\in B) \leqslant p,
	\end{align}
	$f$ fulfills
	\begin{align}
	\mathbb{E} \qty[f(A\cup B)] \geqslant (1-p)f(A)
	\end{align}
	
	\begin{proof}
		Define
		\begin{align}
		g_A(S) = f(A\cup S) 
		\end{align}
		Obviously, $g_A$ is also submodular (from diminishing returns). Then, from \vref{lem:exp_emptyset}
		\begin{align}
		\mathbb{E} \qty[f(A\cup B)] = \mathbb{E} \qty[g(B)]\geqslant (1-p)g(\emptyset) = (1-p)f(A)
		\end{align}
	\end{proof}
\end{lemma}

\begin{theorem}[\citet{buchbinder2014submodular}]
	In non-monotonic case, \vref{rand_greedy} is $\frac{1}{e}$ optimal in expectation.
	\begin{proof}
		Similarly to monotonic case, take a look at $i^{th}$ iteration and condition on previous iterations, denote a chosen element from $M_i$ as $u_i$:
		\begin{align}
		\mathbb{E} \qty\bigg[f(A_{i-1} \cup \qty{u_i}) - f(A_{i-1}) | A_{i-1} ] &= \frac{1}{k} \sum_{u_i \in M_i} f(A_{i-1} \cup \qty{u_i}) - f(A_{i-1}) \geqslant \frac{1}{k} \qty(f(A_{i-1} \cup S^*) - f(A_{i-1}) )
		\end{align}
		
		
		Since
		\begin{align}
		P(u\in A_{i-1}) \leqslant 1 - \qty(1-\frac{1}{k})^{i-1}
		\end{align}
		from \vref{lem:exp}
		\begin{align}
		\mathbb{E}\qty[f(A_{i-1} \cup S^*)] \geqslant \qty(1-\frac{1}{k})^{i-1}f(S^*)
		\end{align}
		Thus, taking expectation		
		\begin{align}
		\mathbb{E} \qty\bigg[f(A_{i-1} \cup \qty{u_i}) - f(A_{i-1}) ] &\geqslant \frac{1}{k} \qty(f(A_{i-1} \cup S^*) - f(A_{i-1}) ) \geqslant \frac{1}{k}  \qty[\qty(1-\frac{1}{k})^{i-1}f(S^*) - \mathbb{E} \qty\bigg[f(A_{i-1})] ]
		\end{align}
			
		\begin{align}
		\mathbb{E} \qty\bigg[f(A_{i} )] &\geqslant \frac{1}{k} \qty(f(A_{i-1} \cup S^*) - f(A_{i-1}) ) \geqslant \frac{1}{k}  \qty[\qty(1-\frac{1}{k})^{i-1}f(S^*) - \mathbb{E} \qty\bigg[f(A_{i-1})] ]
		\end{align}
		Solving the recurrence we get
		\begin{align}
		\mathbb{E} \qty[f(A_i)] \geqslant \frac{i}{k} \qty(1-\frac{1}{k})^{k-1} f(S^*) \geqslant \frac{1}{e}f(S^*) 
		\end{align}
		i.e.,
		\begin{align}
		\mathbb{E} \qty[f(A_k)] \geqslant  \qty(1-\frac{1}{k})^{k-1} f(S^*) \geqslant \frac{1}{e}f(S^*) 
		\end{align}
		
		
	\end{proof}
\end{theorem}

\paragraph{Note} \vref{rand_greedy} is not optimal. In particular, the upper bound of the best approximation is $\sim0.49$.
\paragraph{Runtime} Runtime of \vref{rand_greedy} is $\order{nk}$.

\section{Maximization of the submodular function without constraints}
\begin{align}
\max \: & f(S)
\end{align}
\paragraph{Examples}
\begin{itemize}
\item \textsc{max-cut}
\item \textsc{max-directed-cut}
\item Max Facility Location
\item \textsc{max-sat} (with all literals in a clause having same sign).
\end{itemize}

\begin{prop}[\cite{feige2011maximizing}]
	Algorithm which choose random solution as following: $u\in S$ with probability $\frac{1}{2}$ independently, is $\frac{1}{4}$ approximation in expectation:
	\begin{align}
	\mathbb{E}[f(S)] \geqslant \frac{1}{4} f(S^*)
	\end{align}
\end{prop}
\begin{prop}[\cite{feige2011maximizing}]
If $f$ is symmetric, the same algorithm is $\frac{1}{2}$ approximation in expectation:
\begin{align}
\mathbb{E}[f(S)] \geqslant \frac{1}{2} f(S^*)
\end{align}
\end{prop}
\begin{prop}[\cite{feige2011maximizing}] \label{prop:eps_max}
For any constant $\epsilon>0$ it is impossible to acquire $\qty(\frac{1}{2}+\epsilon)$ approximation in polynomial time, even in  symmetric case.
\end{prop}

Note that for $\bar{f}(S) = f(\bar{S})$, we can use the same oracle. So a ''conjugate`` algorithm would be start from $N$ and drop elements from it.

\paragraph{Note} \cref{prop:eps_max} is only appliciable for our model, where we query the black-box submodular function $f$. A better-than-$\nicefrac{1}{2}$ approximation might be achieved in a model of computation requiring an explicit representation of $f(S)$. For example, best known approximation of \textsc{max-cut} is $0.878$, by using  semidefinite programming \cite{goemans1995improved}.

\begin{algorithm}
	\caption{}\label{alg:double_greedy}
	\begin{algorithmic}[1]
		\Procedure{Double Greedy}{$N$}
		\State $X \gets \emptyset$, $Y \gets N$
		\For{$i=1$ to $n$}
		\State $a_i = f(X_{i-1} \cup \qty{u_i})-f(X_i)$
		\State $b_i = f(Y_{i-1} \setminus \qty{u_i})-f(Y_i)$
		\If{$a_i > b_i$}
		\State $X_i \gets X_{i-1} \cup \qty{u_i}$
		\State $Y_i \gets Y_{i-1}$
		\Else
		\State $X_i \gets X_{i-1}$
		\State $Y_i \gets Y_{i-1} \setminus \qty{u_i}$
		\EndIf
		\EndFor
		\State \textbf{return} $X_N$
		\EndProcedure
	\end{algorithmic}
\end{algorithm}


\begin{algorithm}
	\caption{}\label{alg:rand_double_greedy}
	\begin{algorithmic}[1]
		\Procedure{Randomized Double Greedy}{$N$}
		\State $X \gets \emptyset$, $Y \gets N$
		\For{$i=1$ to $n$}
		\State $a_i = \max \qty{0,f(X_{i-1} \cup \qty{u_i})-f(X_i)}$
		\State $b_i = \max \qty{0,f(Y_{i-1} \setminus \qty{u_i})-f(Y_i)}$
		\State $(X_i,Y_i) \gets\begin{cases}
		\qty(X_{i-1} \cup \qty{u_i}, Y_i) & \text{with } P= \frac{a_i}{a_i+b_i} \\
		\qty(X_{i-1} ,Y_{i-1} \setminus \qty{u_i}) & \text{with } P= \frac{b_i}{a_i+b_i} 
		\end{cases}$
		
		\EndFor
		\State \textbf{return} $X_N$
		\EndProcedure
	\end{algorithmic}
\end{algorithm}