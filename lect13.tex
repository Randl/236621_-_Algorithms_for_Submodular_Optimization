\subsubsection{Submodular set cover }
Given $n$ elements $\qty{1,\dots, n}$ and $m$ subsets $S_1,\dots, S_m \subseteq \qty{1, \dots, n}$ with submodular function $f: 2^{1,\dots, m} \to \mathbb{R}_+$. The goal is to find cover $X\subseteq \qty{S_1, \dots, S_m}$ which minimizes $f(X)$.


\paragraph{Greedy algorithm}
At each step we choose set which minimizes
\begin{align}
V_i(S_j) = \frac{f\qty(X_{i-1} \cup \qty{S_j})-f\qty(X_{i-1} )}{\abs{\bigcup_{k\in X_{i-1}\cup \qty{S_j}} S_k} - \abs{\bigcup_{k\in X_{i-1}} S_k }}
\end{align}
This algorithm gives $\log(n)$ approximation in case of weighted or regular set cover.

\subparagraph{Example}
Let 
\begin{align}
S_j &= \qty{j}\\
S_{n-1} &= \qty{1,2,\dots, n}\\
f(X) &= \min \qty{ \abs{X}, n-\abs{X}, 1}
\end{align}
In this case the optimal solution is $X = \qty{S_1, \dots, S_n}$ while the algorithm will choose $X = \qty{S_{n+1}}$
\paragraph{Frequency}
We define frequency of the elements as 
\begin{align}
\eta_i = \abs{\qty{j: i \in S_j}}
\end{align}
and frequency of input is $\max\limits_i \eta_i$.

Similarly to vertex cover, we'll define relaxed problem
\begin{align} \label{eq:ssp_relax}
\min \quad&  f_L(\vb{x}) \\ 
\st \forall i \in \qty{1,2,\dots, n} \quad&\sum_{j: i\in S_j} x_j \geq 1 \\
\forall \forall i \in \qty{1,2,\dots, n} \quad&0\leq x_i \leq 1 
\end{align}
 and we can find optimal solution $\vb{x}^*$. We define solution in the following way:
 \begin{align}
 X = \qty{S_j: x_j^* \geq \frac{1}{\eta}}
 \end{align}
 
 Obviously the solution is feasible.
 \begin{prop}[\citet{iwata2009submodular}]
 	The solution is $\eta$-approximation
 	\begin{proof}
 		If $f$ is symmetric,
 		\begin{align}
 		f(X) = f_L(\mathds{1}_X) \leq f_L\qty( \min \qty{ \eta \cdot \vb{x}^*, \mathds{1} }) \leq \eta \cdot f_L(\vb{x}^*)
 		\end{align}
 		
 		If $f$ is not monotonic we can ``monotonize'':
 		\begin{align}
 		\tilde{f}(A) = \min \qty\big{f(B): A\subseteq B \subseteq \qty{1, \dots, m}}
 		\end{align}
 		Take a look at
 		\begin{align}
 		\tilde{X} = \arg \min\limits_T \qty{f(T) : X\subseteq T}
 		\end{align}
 		we can use the proof of monotonic case with $\tilde{f}$ and show that the algorithm is $\eta$-approximation even if $f$ is not monotonic.
 	\end{proof}
 \end{prop}