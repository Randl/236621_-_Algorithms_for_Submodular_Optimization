\subsection{Submodular covering problems}
\subsubsection{Submodular vertex cover}
For graph $G=(V,E)$ and submodular $f: 2^V \to \mathbb{R}_+$. The goal is to find $A\subset V$, such that $G$ is vertex cover (i.e., for $ (u,v)\in E$, $u\in A$ or $v\in A$) and $f(A)$.

For each vertex define variable $x_i$, and the relaxation is 
\begin{align} \label{eq:scp_relax}
\min \quad&  f_L(\vb{x}) \\ 
\st \forall (u,v)\in E \quad&x_u+x_v \geqslant 1 \\
\forall u \in V \quad&0\leqslant x_u \leqslant 1 
\end{align}

\begin{prop}[\citet{iwata2009submodular}]
	There exists half-integer optimum of \vref{eq:scp_relax}, i.e., $\forall v\in V$ $x_v^k \in \qty{0,\frac{1}{2}, 1}$. It is possible to find it efficiently.
	\begin{proof}
		Let $\vb{x}^*$ be the optimum of \vref{eq:scp_relax}. Assume 
		\begin{align}
		x^*_{u_1} \geqslant x^*_{u_2} \geqslant \dots \geqslant x^*_{u_n}
		\end{align}
		We can rewrite
		\begin{align}
		f_L(\vb{x}^*) = (1-x^*_{u_1}) \cdot f(\emptyset) + (x^*_{u_1}-x^*_{u_2})f(\qty{u_1}) + \dots + x^*_{u_n} f(N)
		\end{align}
		We can now write optimization problem as a linear programming problem:
		\begin{align} \label{eq:scp_relax_linear}
		\min \quad&  (1-x_{u_1}) \cdot f(\emptyset) + (x_{u_1}-x_{u_2})f(\qty{u_1}) + \dots + x_{u_n} f(N) \\ 
		\st  \quad&x_{u_1} \geqslant x_{u_2} \geqslant \dots \geqslant x_{u_n} \\
		\forall (u,v)\in E \quad&x_u+x_v \geqslant 1 \\
		\forall u \in V \quad&0\leqslant x_u \leqslant 1 
		\end{align}
		Obviously the optimal value of \vref{eq:scp_relax_linear} is less or equal than the optimal value of \vref{eq:scp_relax}.
		
		Let $\vb{x}^*$ be the optimal solution of \vref{eq:scp_relax_linear}. Define
		\begin{align}
		V^- &= \qty{ u : 0 < x^*_u <\frac{1}{2} }\\
		V^+ &= \qty{ u : \frac{1}{2} < x^*_u <1 }\\
		\end{align}
		Now define two new solutions:
		\begin{align}
		y_u &= \begin{cases}
		x^*_u + \epsilon & u \in V^+\\
		x^*_u - \epsilon & u \in V^-\\
		x^*_u & \text{otherwise}\\
		\end{cases}\\
		z_u &= \begin{cases}
		x^*_u - \epsilon & u \in V^+\\
		x^*_u + \epsilon & u \in V^-\\
		x^*_u & \text{otherwise}\\
		\end{cases}
		\end{align}
		Since $\vb{x}^* = \frac{\vb{y}+\vb{z}}{2}$, the value of at least one of $\vb{y}$, $\vb{z}$ is not worse than value of $\vb{x}^*$ (from linearity). In particular, both are $\vb{y}$, $\vb{z}$ have same value as $\vb{x}^*$, since $\vb{x}^*$ is optimal.
		
		We can choose $\epsilon$ such that $y$, $z$ are non-negative and less than $1$.
		
		For edge constrain, it is impossible that $u,v \in V^-$. If $x_u+x_v=1$, both $y$ and $z$ doesn't change value of $x_u+x_v$. If $x_u+x_v>1$, there exists $\epsilon$, such that edge constrain is still fulfilled.
		
		Similarly, order constraints are fulfilled for small enough $\epsilon$.
		
		Since $\vb{y}$ moves values towards $0$ and $1$ and $\vb{z}$, we choose feasible $\epsilon$ such that one of values in $\vb{y}$ or $\vb{z}$ becomes half-integer, thus increasing number of half-integer elements in optimal solution, until we arrive to half-integer solution.
	\end{proof}
\end{prop}

\begin{prop}[\citet{iwata2009submodular}]
	Given half-integer optimal solution, 
	\begin{align}
	A = \qty{ u \big\vert x^*_u \geqslant \frac{1}{2} }
	\end{align}
	is 2-approximation of submodular vertex cover.
	\begin{proof}
		\begin{align}
			f_L(\vb{x}^*) = \frac{1}{2} f(A) + \frac{1}{2} f\qty(\qty{u|x^*_u=1}) \leqslant  \frac{1}{2} f(A)
		\end{align}
	\end{proof}
\end{prop}

\begin{prop}[\citet{khot2008vertex}]
	Under unique games conjecture, 2-approximation is optimal.
\end{prop}